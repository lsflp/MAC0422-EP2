\documentclass[]{scrartcl}
\usepackage[utf8]{inputenc}
\usepackage[portuguese]{babel}


%opening
\title{EP2 - MAC0422}
\author{}

\begin{document}

\maketitle 

\begin{abstract}

\end{abstract}

\section{Como vai funcionar!}

Nesse EP vamos querer facilitar a nossa vida. Nele vamos colocar criar dois vetores de tamanho $2*d$. Nele, cada ciclista vai ocupar duas posições, pois cada ciclista tem tamanho 1 metro e cada elemento do nosso vetor representa 0.5m. Não optamos por usar um vetor de tamanho $d$, pois se um ciclista se mover a 0,5m em um passo da simulação de 60ms.

Vão existir 4 vetores de pista, cada um com tamanho $2*n$:

\begin{enumerate}
	\item Pista interna $(round = 0)$
	\item Pista interna $(round = 1)$
	\item Pista externa $(round = 0)$
	\item Pista externa $(round = 1)$
\end{enumerate}

A cada passo da simulação existirá uma variável $round$, ela indicará para qual pista o ciclista deve ir. Os passos da nosso programa serão dessa maneira: cada ciclista vai para a pista interna do $round$ diferente do atual, se a posição de destino já estiver ocupada tenta ir para a pista externa (fazer ultrapassagem), caso a externa esteja ocupada, vai ficar atrás; as threads são sincronizadas e uma delas verifica as condições de termino e aleatoriedade da corrida; repetimos os passos.


\section{Considerações}
Para um ciclista sair da pista mais interna e ir para a externa, ultrapassar, é o mesmo custo dele ir para frente.

Cada ciclista é autônomo, ele sabe as regras e ele decide se vai ultrapassar ou não.

A corrida começa com todos os ciclistas fora da pista. E a cada passo da simulação os ciclistas da equipe $A$ são inseridos na posição 0 e 1 do vetor e os ciclistas da equipe $B$ são inseridos na posição $d$ e $d+1$. Ele somente consegue entrar na pista interna.

A questão do ciclista contido (o que tem que ficar a 30km/h) vai funcionar da seguinte maneira: quando um ciclista passar na linha de chegada, ele vai ver se existe algum colega com numero de voltas maior do que o dele em sua frente cujo tenha a velocidade de 30km/h, se ninguém tiver ele sortea a velocidade, se alguém tiver ele vai andar a 30 km/h!.

Note que um time inicia na posição 0 e o outro na posição d/2!


\section{Barreiras}
	Nos implementamos com ...
	
\section{Probabilidade de quebra}
	Temos no enunciado que a probabilidade de quebra é $10\%$ e 4 rodadas. Vamos considerar que ela seja uniforme, ou seja, a probabilidade de quebrar em movimento será:
	
	$$
	P(q) = \frac{10\% * v}{d*8}
	$$
	Onde $v$ é a velocidade e representa 1 ou 2, sendo 1 para 30 km/h e 2 para 60 km/h. O Exemplo: suponha que o ciclista estaja andando a 60km/h, a probabilidade de ele quebrar a cada paço da interação será em uma posta de 400 metros será  0.00125.
	
	Note que o ciclista vai andar e quebrar, ou seja, ele não pode começar quebrado, mas ele pode chegar quebrado na linha de chegada.
	
	Veja que no código utilizamos o tempo como raiz da função randómica. Além disso temos a seguinte fórmula:
	
	$$
	P(q) = \frac{speed}{80*d}
	$$ 
	
	Note que $2 * d_{MAX} = pos_{MAX}$, com isso podemos observar da onde vem o 40 dessa fórmula. 


\section{Argumentos}

\begin{center}
	$ep2$ $d$ $n$ $[v|u]$ $[-v]$
\end{center}

$d$ numero inteiro, comprimento em metros da pista.

$n$ número de ciclista em cada equipe.

$v$ usado para definir simulações com velocidades aleatórias a cada volta. 

$u$ usado para definir simulações com velocidades uniformes de 60Km/h.

$-v$ opção para mostrar tudo, utilizado para produzir saída para a interface gráfica.

\end{document}
